\documentclass{beamer}
\usetheme{Madrid}
\usecolortheme{crane}

\usepackage{fontspec, xeCJK}
\usefonttheme{professionalfonts}
\setsansfont{Andika}[Renderer=Harfbuzz]
\setCJKmainfont{Noto Sans CJK SC}[Renderer=Harfbuzz]
% \usefonttheme{professionalfonts}\usefonttheme{serif}
% \setmainfont{Gentium Plus}
% \setCJKmainfont{Noto Sans CJK SC}


% \setmainfont{Linux Libertine O}[
%   Mapping=tex-text,
%   Renderer=Harfbuzz,
%   FallbackFonts = {
%     { Name = Charis SIL,
%       Range = {
%         "IPA Extensions",
%         "Phonetic Extensions",
%         "Phonetic Extensions Supplement",
%         "Spacing Modifier Letters",
%         "Latin Extended-B",
%         "Latin Extended Additional"
%       }
%     }
%   }
% ]

\usepackage{gb4e}

% NOTE: Add your preferred fonts here, e.g.
% \setsansfont{TeX Gyre Heros}
% \setmainfont{TeX Gyre Termes}

\title{Soundscapes of East and Southeast Asia}
\author{Francis Bond}
\date{}

\newcommand{\booksrc}[1]{\hfill {\tiny [#1]}}
%\newcommand{\booksrc}[1]{}

\begin{document}
\frame{\titlepage}

% ======================================================
% Section 1: What the chapter covers
% ======================================================

\begin{frame}{Chapter Map}
\begin{itemize}
\item Phoneme systems: size, contrasts, analytic controversies \booksrc{5.1}
\item Word shapes (phonotactics) across regions \booksrc{5.2}
\item Tones and allotones; tone sandhi typology \booksrc{5.3}
\item Shifting sounds (morphophonemics) \booksrc{5.4}
\item Pitch-accent in Japanese \booksrc{5.5}
\end{itemize}
\end{frame}

\begin{frame}{A Word of Caution: Counting Phonemes}
\begin{itemize}
\item Identifying phonemes is often non-trivial \booksrc{5.1}
\item Analysts disagree on how many phonemes a language has \booksrc{5.1}
\item Same sound, different symbols across traditions \booksrc{5.1}
\item Orthography vs. IPA: frequent mismatches \booksrc{5.1}
\item Balance between phonetic detail and phonological economy \booksrc{5.1}
\end{itemize}
\end{frame}

\begin{frame}{Regional Big Picture}
\begin{itemize}
\item Insular SE Asia: smaller systems, many disyllables \booksrc{5.1.1, 5.2.1}
\item Mainland SE Asia: larger systems, aspiration contrasts \booksrc{5.1.2}
\item Sinitic: rich initials/finals, tone, restricted codas \booksrc{5.2.2}
\item Korean/Japanese: special laryngeal or prosodic systems \booksrc{5.1.3, 5.5}
\item Tone vs. pitch-accent: different organizing principles \booksrc{5.3, 5.5}
\end{itemize}
\end{frame}

% ======================================================
% Section 2: Insular Southeast Asia (Austronesian)
% ======================================================

\begin{frame}{Malay Consonants (Indigenous Set)}
\begin{tabular}{lllll}
 & Labial & Alveolar & Palatal & Velar \\
\hline
Stops & p, b & t, d & c, j & k, g \\
Nasals & m & n & ny [ɲ] & ng [ŋ] \\
Fricatives & s &  &  &  \\
Glottal &  &  &  & [ʔ] (alloph.) \\
Lateral &  & l &  &  \\
Trill &  & r &  &  \\
Semivowels & w & y [j] &  &  \\
\end{tabular}
\begin{itemize}
\item 18 indigenous consonants; [ʔ] historically an allophone of /k/ \booksrc{5.1.1}
\item Loans add f, v, z, sy[ʃ], kh[x] (educated speech) \booksrc{5.1.1}
\item Digraphs = single phonemes (ny, ng, sy, kh) \booksrc{5.1.1}
\item Orthography ≠ IPA in some cases \booksrc{5.1.1}
\item % illustration: table with Malay orthography vs IPA
\end{itemize}
\end{frame}

\begin{frame}{Malay: Glottal Stop vs. /k/ in Loans}
\begin{exe}
\ex kakak [kakaʔ] \quad ‘older sister’ \\
\ex rakyat [raʔjat] \quad ‘citizenry’ \\
\ex fisik [fisik] \quad vs.\ bisik [bisiʔ] \\
\trans allophonic [ʔ] vs. emergent contrast in loans
\end{exe}
\begin{itemize}
\item Word-final /k/ → [ʔ] (traditional/village Malay) \booksrc{5.1.1}
\item Educated speech: contrast with [ʔ] in loans \booksrc{5.1.1}
\item Orthography keeps “k” for both, masking contrast \booksrc{5.1.1}
\item Pathway: allophone → phoneme through borrowing \booksrc{5.1.1}
\item % illustration: waveform/spectrogram of [k] vs [ʔ]
\end{itemize}
\end{frame}

\begin{frame}{Tagalog vs. Malay: Vowel Systems}
\begin{tabular}{lll}
Tagalog & Malay & Notes \\
\hline
i, u & i, u & high vowels \\
e, o & e, o & mid vowels \\
a & a & low vowel \\
--- & ə (or ɨ/ɜ) & central vowel (Malay) \\
\end{tabular}
\begin{itemize}
\item Tagalog: 5 vowels; Malay: 6 (tense central vowel) \booksrc{5.1.1}
\item Symbol choice debated: ə/ɨ/ɜ/ɘ for Malay central vowel \booksrc{5.1.1, 5.1.1 tbls.}
\item Total phonemes (incl. loans) still < 30 \booksrc{5.1.1}
\item Austronesian systems often compact \booksrc{5.1.1}
\item % illustration: vowel chart overlay (Tagalog vs. Malay)
\end{itemize}
\end{frame}

\begin{frame}{Acehnese Vowels (Sumatra)}
\begin{tabular}{lll}
Front & Back-central (unrounded) & Back (rounded) \\
\hline
i & ɯ & u \\
e & ə & o \\
ɛ & ʊ & ɔ \\
a &  &  \\
\end{tabular}
\begin{itemize}
\item Ten simple vowels + diphthongs \booksrc{5.1.1}
\item Typologically closer to Mon-Khmer \booksrc{5.1.1}
\item Shows diversity within Austronesian \booksrc{5.1.1}
\item Suggests contact/areal effects in vowel space \booksrc{5.1.1}
\item % illustration: Acehnese vowel trapezoid
\end{itemize}
\end{frame}

\begin{frame}{Insular SE Asia: Word Shapes (CVCV Bias)}
\begin{itemize}
\item Western side (Malay/Tagalog): many disyllabic lexemes \booksrc{5.2.1}
\item Template: (C)V\textsubscript{1}(C)(C)V\textsubscript{2}(C) (idealized) \booksrc{5.2.1}
\item Monosyllables: mostly exclamations/clippings \booksrc{5.2.1}
\item Polysyllables: compounds/derivations; some loans \booksrc{5.2.1}
\item East Indonesia: simpler CVCV roots preferred \booksrc{5.2.1}
\end{itemize}
\end{frame}

\begin{frame}{Insular SE Asia: V\textsubscript{2} Neutralizations}
\begin{itemize}
\item In Malay: final open syllables prefer high vowels \booksrc{5.2.1}
\item Closed finals prefer mid vowels (e/o) \booksrc{5.2.1}
\item Some dialects: final continuants lost; only [ʔ] as obstruent \booksrc{5.2.1}
\item Final nasals neutralized to [ŋ] or nasalization \booksrc{5.2.1}
\item % illustration: dialectal maps of Peninsular Malay finals
\end{itemize}
\end{frame}

% ======================================================
% Section 3: Mainland Southeast Asia
% ======================================================

\begin{frame}{Mainland SE Asia: Segmental Profiles}
\begin{itemize}
\item Typically 30–40+ phonemes \booksrc{5.1.2}
\item Voicing often non-distinctive; aspiration is \booksrc{5.1.2}
\item Two (or three) stop series; stiff/creaky series occur \booksrc{5.1.2, 5.3.1}
\item Vowels: larger systems (Mon-Khmer especially) \booksrc{5.1.2}
\item Tone often linked to earlier consonant distinctions (tonogenesis) \booksrc{5.3.1}
\end{itemize}
\end{frame}

\begin{frame}{Thai Consonants (Hudak 1987)}
\begin{tabular}{lllll}
 & Labial & Alveolar & Palatal & Velar \\
\hline
Stops (UA) & p & t & c & k \\
Stops (A) & ph & th & ch & kh \\
Stops (V) & b & d &  &  \\
Nasals & m & n &  & ŋ \\
Fricatives & f & s &  & h \\
Rhotic & r &  &  &  \\
Lateral & l &  &  &  \\
Semivowels & w & y &  &  \\
\end{tabular}
\begin{itemize}
\item 20 consonant phonemes \booksrc{5.1.2 tbl.}
\item Aspiration contrasts; ‘‘voiced’’ often stiff-voiced \booksrc{5.1.2}
\item Common romanization ≠ strict IPA \booksrc{5.1.2}
\item Unrounded back vowels in the system \booksrc{5.1.2}
\item % illustration: Thai script–IPA–romanization alignment
\end{itemize}
\end{frame}

\begin{frame}{Thai Vowels (Hudak 1987)}
\begin{tabular}{lll}
Front & Central & Back \\
\hline
i, e, ɛ & a, ə & ɯ, u, o, ɔ \\
\end{tabular}
\begin{itemize}
\item ~9 vowels; length contrasts important \booksrc{5.1.2}
\item Some symbols used non-IPA in romanization \booksrc{5.1.2}
\item Dense mid space; phonotactic distributional limits \booksrc{5.3.2}
\item Contours avoid short vowels (timing) \booksrc{5.3.2}
\item % illustration: Thai vowel chart with length marks
\end{itemize}
\end{frame}

\begin{frame}{Vietnamese Overview}
\begin{itemize}
\item ~23 consonants; 12 vowels \booksrc{5.1.2}
\item Voiced fricatives distinctive; retroflexes present \booksrc{5.1.2}
\item Extra low front-central vowels \booksrc{5.1.2}
\item Tone system: multiple contours; rich diacritics \booksrc{5.3}
\item Phonotactics: (C)(w)V(V)(C); limited codas \booksrc{5.2.2}
\end{itemize}
\end{frame}

\begin{frame}{Burmese Consonants (Large Set)}
\begin{itemize}
\item ~34 consonants (3 marginal) \booksrc{5.1.2}
\item Three stop series: plain, aspirated, voiced \booksrc{5.1.2}
\item Voiceless nasals (hm, hn, hŋ) \booksrc{5.1.2}
\item Voiceless approximants (hw, hl) \booksrc{5.1.2}
\item Sesquisyllabic: C\textsubscript{1}ə – C\textsubscript{1}(w/y)V(C\textsubscript{2}) \booksrc{5.2.2}
\end{itemize}
\end{frame}

\begin{frame}{Lahu and Related Tibeto-Burman}
\begin{itemize}
\item Lahu: lacks voiceless nasals; adds post-velar stops \booksrc{5.1.2}
\item Strikingly lacks /s/ \booksrc{5.1.2}
\item Considerable variation across TB languages \booksrc{5.1.2}
\item Tone present; sandhi patterns vary \booksrc{5.3}
\item % illustration: feature inventory heatmap
\end{itemize}
\end{frame}

\begin{frame}{Khmer and Mon-Khmer Vowels}
\begin{itemize}
\item Khmer (Battambang): 27 vocalic nuclei \booksrc{5.1.2}
\item Length, diphthongs, register remnants \booksrc{5.1.2}
\item Bru reported: up to 68 vocalic nuclei \booksrc{5.1.2}
\item Neutralization patterns in minor syllables \booksrc{5.2.2}
\item % illustration: Khmer register → length diachrony
\end{itemize}
\end{frame}

\begin{frame}{Hmong-Mien: Very Large Inventories}
\begin{itemize}
\item Four stop series (plain/prenasalized × unasp/asp) \booksrc{5.1.2}
\item Six places of articulation; many affricates \booksrc{5.1.2}
\item Voiced/voiceless nasals; lateral releases \booksrc{5.1.2}
\item Inventory size among world’s largest \booksrc{5.1.2}
\item Tone contrasts layered atop segmental richness \booksrc{5.3}
\end{itemize}
\end{frame}

% ======================================================
% Section 4: Sinitic Languages
% ======================================================

\begin{frame}{Mandarin: Initials and Finals}
\begin{itemize}
\item 20 consonant initials; aspiration contrasts \booksrc{5.1.3}
\item Retroflex fricatives/affricates; palatals allophonic \booksrc{5.1.3}
\item Finals (rhymes) include many diphthongs/triphthongs \booksrc{5.2.2}
\item Only /n, ŋ/ as codas; no clusters \booksrc{5.2.2}
\item Pinyin standard for romanization \booksrc{5.1.3}
\end{itemize}
\end{frame}

\begin{frame}{Mandarin Finals (Beijing)}
\begin{itemize}
\item Rich set of rhymes: i, u, ü; -ai, -ei, -ao, -ou; -an, -en, -ang, -eng; -ong \booksrc{5.2.2}
\item Triphthongs like -iao, -ian, -iong \booksrc{5.2.2}
\item Only nasals as codas; no clusters after triphthongs \booksrc{5.2.2}
\item Neutral tone in weak syllables (e.g., \emph{de}) \booksrc{5.3.2}
\item % illustration: Mandarin initial–final chart
\end{itemize}
\end{frame}

\begin{frame}{Cantonese: Initials and Tones}
\begin{tabular}{lllll}
 & Labial & Dental/Alv. & Palatal & Velar \\
\hline
Stops UA & b & d &  & g \\
Stops A & pʰ & tʰ & tsʰ & kʰ \\
Affricates & ts & tsʰ &  &  \\
Fricatives & f & s &  & h \\
Nasals & m & n &  & ŋ \\
Glide & w & j &  &  \\
\end{tabular}
\begin{itemize}
\item 6 lexical tones; sometimes “9” including checked \booksrc{5.1.3, 5.3.2}
\item Checked tones are allotones (shortened level tones) \booksrc{5.3.2}
\item No tone sandhi proper (phonological) \booksrc{5.3.3}
\item Tone-bearing unit: syllable/morpheme (E/SEA style) \booksrc{5.3.4}
\item % illustration: Cantonese level vs. checked tone contours
\end{itemize}
\end{frame}

% ======================================================
% Section 5: Korean & Japanese
% ======================================================

\begin{frame}{Korean: Laryngeal Three-way Stops}
\begin{itemize}
\item Aspirated, lax, tense series \booksrc{5.1.3}
\item Tense: glottalized/high muscular tension \booksrc{5.1.3}
\item Lax: voiceless unaspirated (often perceived as voiced) \booksrc{5.1.3}
\item Neutralization and allophony (e.g., /r/ ~ [l]/[r]) \booksrc{5.1.3}
\item Morphophonemics: many alternations \booksrc{5.4.1}
\end{itemize}
\end{frame}

\begin{frame}{Korean Vowels (Simple Set)}
\begin{tabular}{lll}
Front & Central & Back \\
\hline
i, e, ae & ʌ & ɯ, u, o, a \\
\end{tabular}
\begin{itemize}
\item ~8 simple vowels (front rounded often diphthongs) \booksrc{5.1.3}
\item High back unrounded [ɯ] notable \booksrc{5.1.3}
\item High vowels often devoiced between voiceless Cs \booksrc{5.1.3}
\item Complex vowels arise in suffixation contexts \booksrc{5.1.3, 5.4.1}
\item % illustration: devoicing spectrogram (ki̥tsɯ)
\end{itemize}
\end{frame}

\begin{frame}{Japanese: Small Segmental System}
\begin{itemize}
\item 5-vowel system (Tagalog-like configuration) \booksrc{5.1.3}
\item Simple consonant inventory; few places of articulation \booksrc{5.1.3}
\item High vowels devoiced in voiceless environments \booksrc{5.1.3}
\item Mora as rhythmic unit (N and Q as moraic consonants) \booksrc{5.2.2, 5.4.3}
\item Pitch-accent organizes lexical contrasts \booksrc{5.5}
\end{itemize}
\end{frame}

\begin{frame}{Japanese Mora in Examples}
\begin{exe}
\ex hachimaki \quad ha--chi--ma--ki (4 moras) \\
\ex shinbun \quad shi--N--bu--N (4 moras) \\
\ex yappari \quad ya--Q--pa--ri (4 moras)
\end{exe}
\begin{itemize}
\item Mora ≈ timing unit; poetry and kana reflect moraic counts \booksrc{5.2.2}
\item N = moraic nasal; Q = underspecified moraic obstruent \booksrc{5.2.2}
\item Geminate clusters: first consonant counts as a mora \booksrc{5.4.3}
\item Place-assimilatory realization of N before obstruents \booksrc{5.2.2}
\item % illustration: mora timing diagram
\end{itemize}
\end{frame}

% ======================================================
% Section 6: Phonotactics (Comparative)
% ======================================================

\begin{frame}{Mainland vs. Insular: Macro-Contrast}
\begin{itemize}
\item Mainland: monosyllables; sesquisyllables (Mon-Khmer) \booksrc{5.2.2}
\item Insular: disyllabic roots; CVCV preference eastward \booksrc{5.2.1}
\item Sinitic: initial + rhyme (“final”); restricted codas \booksrc{5.2.2}
\item Korean: clusters allowed; simplifications final \booksrc{5.2.2}
\item Japanese: strict CV; only N and Q codas \booksrc{5.2.2}
\end{itemize}
\end{frame}

\begin{frame}{Mon-Khmer Sesquisyllables}
\begin{itemize}
\item ‘‘One-and-a-half’’ syllables: minor + major \booksrc{5.2.2}
\item Minor: unstressed; reduced vowel inventory \booksrc{5.2.2}
\item Khmer: C(r)V\textsubscript{1}(N) – C(C)V\textsubscript{2}(C) \booksrc{5.2.2}
\item Rich clusters in major syllables; unusual onsets \booksrc{5.2.2}
\item Burmese adopted sesquisyllabic pattern historically \booksrc{5.2.2}
\end{itemize}
\end{frame}

\begin{frame}{Chinese Syllable Structure (Mandarin)}
\begin{itemize}
\item Template: (C)(V)V(V/N) \booksrc{5.2.2}
\item Diphthongs/triphthongs common in nucleus \booksrc{5.2.2}
\item Only /n, ŋ/ codas; no clusters \booksrc{5.2.2}
\item Analysis by “finals” (rhymes) traditional \booksrc{5.2.2}
\item % illustration: Initials × Finals grid
\end{itemize}
\end{frame}

\begin{frame}{Vietnamese Syllable Template}
\begin{itemize}
\item (C)(w)V(V)(C) \booksrc{5.2.2}
\item Codas: nasal or voiceless stop \booksrc{5.2.2}
\item On-glides (w) frequent; many VV combinations \booksrc{5.2.2}
\item Tone interacts with syllable type \booksrc{5.3.2}
\item % illustration: Vietnamese vowel sequences with allowable codas
\end{itemize}
\end{frame}

\begin{frame}{Korean vs. Japanese: Final Restrictions}
\begin{itemize}
\item Korean allows codas/clusters but neutralizes \booksrc{5.2.2}
\item Cluster simplification in morphological contexts \booksrc{5.4.1}
\item Japanese: only N or first half of geminate allowed \booksrc{5.2.2, 5.4.3}
\item Heavy adaptation of complex-loan syllables \booksrc{5.2.2}
\item % illustration: loanword adaptation flow (EN→JA)
\end{itemize}
\end{frame}

% ======================================================
% Section 7: Tone, Allotone, Sandhi
% ======================================================

\begin{frame}{Phonetic Foundations of Tone}
\begin{itemize}
\item Pitch = perceptual correlate of F0 \booksrc{5.3.1}
\item Controlled via laryngeal tension + airflow \booksrc{5.3.1}
\item Voice qualities: stiff, creaky, breathy co-vary with tone \booksrc{5.3.1}
\item Stiff/creaky often with low tones (e.g., Cantonese) \booksrc{5.3.1}
\item Tonogenesis: consonant contrasts → pitch contrasts \booksrc{5.3.1}
\end{itemize}
\end{frame}

\begin{frame}{Thai: Five Tone Categories}
\begin{exe}
\ex khaa (mid) ‘to be lodged in’ \\
\ex khàa (low) ‘aromatic root’ \\
\ex khâa (falling) ‘servant’ \\
\ex kháa (high) ‘to trade’ \\
\ex khǎa (rising) ‘leg’
\end{exe}
\begin{itemize}
\item Distribution constrained by syllable type \booksrc{5.3.2}
\item Short vowels resist complex contours \booksrc{5.3.2}
\item Checked syllables restrict tonal options \booksrc{5.3.2}
\item No neutral tone (contrast with Mandarin) \booksrc{5.3.2}
\item % illustration: Thai pitch tracks by syllable type
\end{itemize}
\end{frame}

\begin{frame}{Allotones: Mandarin \& Cantonese}
\begin{itemize}
\item Mandarin 3rd tone: 214 → 21 before another tone \booksrc{5.3.2}
\item “Neutral tone” in weak syllables (no inherent tone) \booksrc{5.3.2}
\item Cantonese “9 tones” vs “6”: checked tones are allotonic \booksrc{5.3.2}
\item Checked tones = shortened level tones (H/M/L) \booksrc{5.3.2}
\item Lao dialect: 33 vs 213 in complementary distribution \booksrc{5.3.2}
\end{itemize}
\end{frame}

\begin{frame}{Tone Sandhi: Mandarin}
\begin{itemize}
\item 35 after 35/55 → 55 (assimilatory) \booksrc{5.3.3}
\item 214 + 214 → 35 + 214 (dissimilatory) \booksrc{5.3.3}
\item Shorthand: 35/55 35 ⇒ 35/55 55; 214 214 ⇒ 35 214 \booksrc{5.3.3}
\item Many dialects: additional/local sandhi patterns \booksrc{5.3.3}
\item % illustration: stepwise tone sandhi schematics
\end{itemize}
\end{frame}

\begin{frame}{Tone Sandhi: Tianjin vs. Cantonese}
\begin{itemize}
\item Tianjin (21,45,13,53) rules: \booksrc{5.3.3}
\item 21+21→13+21; 13+13→45+13; 53+53→21+53; 53+21→45+21 \booksrc{5.3.3}
\item Cantonese: little/no phonological sandhi \booksrc{5.3.3}
\item Morphology (e.g., reduplication) can alter tone \booksrc{5.3.3}
\item Allotones vs. sandhi: keep notions distinct \booksrc{5.3.2–3}
\end{itemize}
\end{frame}

\begin{frame}{Tonal Phonology in Broader Perspective}
\begin{itemize}
\item ~1/3 of world’s languages have tone \booksrc{5.3.4}
\item African/Mesoamerican vs. E/SE Asian tone systems \booksrc{5.3.4}
\item E/SEA: more contour tones; syllable-level bearing \booksrc{5.3.4}
\item Autosegmental phonology informed by tone \booksrc{5.3.4}
\item Feature analyses: contours as level combinations \booksrc{5.3.4}
\end{itemize}
\end{frame}

% ======================================================
% Section 8: Morphophonemics
% ======================================================

\begin{frame}{Morphophonemics: General Idea}
\begin{itemize}
\item Morphemes change shape in context \booksrc{5.4}
\item Underlying form + context-sensitive rules \booksrc{5.4}
\item English plural: /-s/, /-z/, /-əz/; voicing + epenthesis \booksrc{5.4}
\item Rules: assimilation, deletion, insertion, coalescence \booksrc{5.4}
\item Feature-copying as common mechanism \booksrc{5.4}
\end{itemize}
\end{frame}

\begin{frame}{Korean: When Vowels Collide}
\begin{itemize}
\item Complex vowels often arise in suffixation \booksrc{5.4.1}
\item Vowel coalescence: o+i→oe; eo+i→e; a+i→ae; u+i→wi \booksrc{5.4.1}
\item Consonant cluster simplifications in morphology \booksrc{5.4.1}
\item Stop→nasal before nasal in compounding \booksrc{5.4.1}
\item Multiple rules can apply sequentially \booksrc{5.4.1}
\end{itemize}
\end{frame}

\begin{frame}{Korean: Cluster and Liquid Alternations}
\begin{itemize}
\item Second obstruent in cluster becomes tense (e.g., ip-da→iptta) \booksrc{5.4.1}
\item /r/→/n/ unless following a vowel \booksrc{5.4.1}
\item Chains: /r/→/n/ then stop→nasal assimilation \booksrc{5.4.1}
\item Example: baeg-ri→baengni; sueobryo→sueomnyo \booksrc{5.4.1}
\item Lexical strata: rule application can be stratum-sensitive \booksrc{5.4.1}
\end{itemize}
\end{frame}

\begin{frame}{Austronesian: Nasal Alternation}
\begin{itemize}
\item Prefixal N- realizes as homorganic nasal \booksrc{5.4.2}
\item Interaction with base-initial consonant \booksrc{5.4.2}
\item Productive in Malay/Tagalog derivation \booksrc{5.4.2}
\item Reflects phonotactic legality + assimilation \booksrc{5.4.2}
\item % illustration: table of N- allomorphs by following C
\end{itemize}
\end{frame}

\begin{frame}{Japanese: Q and N in Morphology}
\begin{exe}
\ex tat-i + -ta → tatta ‘stood’ \\
\ex tor-i + -ta → totta ‘took’ \\
\ex kaw-i + -ta → katta ‘bought’ \\
\ex hati + seN → hasseẽ ‘eight thousand’
\end{exe}
\begin{itemize}
\item Traditional view: Q = underspecified moraic C \booksrc{5.4.3}
\item Orthography (kana) writes Q and N as letters \booksrc{5.4.3}
\item Phonological analysis vs. spelling conventions \booksrc{5.4.3}
\item N assimilation; word-final N ~ uvular nasal in careful speech \booksrc{5.2.2}
\item % illustration: kana mapping of Q/N sequences
\end{itemize}
\end{frame}

% ======================================================
% Section 9: Pitch-Accent in Japanese
% ======================================================

\begin{frame}{Pitch-Accent: Core Properties}
\begin{itemize}
\item Words have characteristic pitch patterns \booksrc{5.5}
\item Functional load lower than in tonal languages \booksrc{5.5}
\item Domain is the word; not per-syllable tones \booksrc{5.5}
\item Predictable once accent position is known \booksrc{5.5}
\item Tokyo vs. Kyoto differences in patterns \booksrc{5.5}
\end{itemize}
\end{frame}

\begin{frame}{Minimal Pairs by Pitch (Tokyo vs. Kyoto)}
\begin{exe}
\ex Tokyo: \textit{ame} (LH) ‘candy’ vs. \textit{ame} (HL) ‘rain’ \\
\ex Kyoto/Tokyo: \textit{hashi} contrasts: ‘edge’, ‘chopsticks’, ‘bridge’
\end{exe}
\begin{itemize}
\item Few minimal pairs overall \booksrc{5.5}
\item Accent position determines post-accent L plateau \booksrc{5.5}
\item Pre-accent H with initial L exception rule \booksrc{5.5}
\item Syllable vs. mora considerations (simplified here) \booksrc{5.5}
\item % illustration: HL/LH curves over multi-syllabic words
\end{itemize}
\end{frame}

\begin{frame}{Tokyo Patterns Worked Example}
\begin{exe}
\ex No accent: \textit{ame} LH; \textit{sakura} LHH; \textit{shirakaba} LHHH \\
\ex Accent: \textit{a’me} HL; \textit{za’kuro} HLL; \textit{ka’makiri} HLLL \\
\ex Pre-accent rule: second syllable H ⇒ initial becomes L
\end{exe}
\begin{itemize}
\item Accented syllable = H; following syllables L \booksrc{5.5}
\item If 2nd syllable is H, 1st becomes L (initial lowering) \booksrc{5.5}
\item Otherwise pre-accent syllables are H \booksrc{5.5}
\item Mora vs. syllable refinements omitted (intro level) \booksrc{5.5}
\item % illustration: step graph of L/H sequences by accent site
\end{itemize}
\end{frame}

% ======================================================
% Section 10: Wrap-up & Study Aids
% ======================================================

\begin{frame}{Key Takeaways}
\begin{itemize}
\item Insular vs. mainland profiles: size, syllables, tones \booksrc{5.1–5.3}
\item Sinitic: initials/finals + tones; limited codas \booksrc{5.2–5.3}
\item Korean: rich morphophonemics; three-way stops \booksrc{5.1.3, 5.4.1}
\item Japanese: mora-timed, N/Q, pitch-accent \booksrc{5.2.2, 5.4.3, 5.5}
\item Allotone vs. sandhi: environment vs. tonal change \booksrc{5.3.2–3}
\end{itemize}
\end{frame}

\begin{frame}{Suggested Illustrations (insert later)}
\begin{itemize}
\item Maps: Insular vs. Mainland SE Asia, language families
\item Spectrograms: [k] vs. [ʔ]; devoiced high vowels (JA/KR)
\item Pitch plots: Thai 5 tones; Mandarin 3rd-tone allotone
\item Vowel charts: Tagalog/Malay/Acehnese; Thai; Khmer
\item Phonotactic schematics: sesquisyllables; JA mora timing
\end{itemize}
\end{frame}

\begin{frame}{Practice: Identify Processes (gb4e prompts)}
\begin{exe}
\ex Malay: \textit{N-} + base-initials (fill the table) \\
\ex Korean: derive surface forms across two rules \\
\ex Mandarin: apply 3rd-tone allotone and sandhi \\
\ex Thai: assign tones by syllable type constraints \\
\ex Japanese: predict pitch from accent position
\end{exe}
\begin{itemize}
\item Turn these into short worksheet items
\item Encourage students to annotate IPA + tone/HL
\item Use Praat screenshots for F0 evidence
\item Mix recognition + production tasks
\item Tie to term project (mini phonology of one language)
\end{itemize}
\end{frame}

\frame{\centering \Huge Questions?}

\end{document}
