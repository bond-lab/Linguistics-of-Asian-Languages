\documentclass{beamer}
\usepackage{fontspec}
\usepackage{xeCJK}
\setCJKmainfont{Noto Sans CJK SC}
\usetheme{Madrid}
\usecolortheme{crane}

\usepackage{expex}

\title{Characteristics of East and Southeast Asian Languages}
\author{Based on Goddard 2005}
\date{}

\begin{document}

\frame{\titlepage}

\begin{frame}
\frametitle{Overview}
\begin{itemize}
\item Lack of inflection (izolační charakter slovotvorby a tvarosloví)
\item Word order (slovosled)
\item Lexical tone (významotvorný tón)
\item Classifier constructions (numerátory)
\item Serial verb constructions (serializace sloves)
\item Multiple pronouns and other systems of address
\item Honorific forms
\end{itemize}
\end{frame}

\section{Lack of Inflection}

\begin{frame}
\frametitle{Lack of Inflection (izolační charakter slovotvorby a tvarosloví)}
\begin{itemize}
\item Many East and Southeast Asian languages lack inflection
\item Words do not change form to indicate:
  \begin{itemize}
  \item Tense
  \item Aspect
  \item Grammatical roles
  \end{itemize}
\item Examples: Malay (Austronesian), Thai (Tai-Kadai)
\item Contrast: Korean (agglutinative morphology)
\end{itemize}
\end{frame}

\begin{frame}
\frametitle{Example: Lack of Inflection in Malay}
\begin{exampleblock}{Malay (Austronesian)}
  \pex
  \a \begingl
\gla Awak marah pada saya ke? //
\glb you angry at I \textsc{ques} //
\glft `Are you angry at me?' //
\endgl

\a \begingl
\gla Ye, memang saya marah pada awak. //
\glb yes certainly I angry at you //
\glft `Yes, I certainly am angry at you.' //
\endgl
\xe
\end{exampleblock}
\begin{itemize}
\item No change in pronoun forms (saya 'I' vs. English 'I/me')
\item No verb inflection for tense or person
\end{itemize}
\end{frame}

\begin{frame}
\frametitle{Example: Inflection in Korean (for contrast)}
\begin{exampleblock}{Korean (Koreanic)}
  \ex
\begingl
\gla Ka-si-ess-keyss-sup-ni-ta //
\glb go-\textsc{hon}-\textsc{pst}-\textsc{pres}-\textsc{pol}-\textsc{ind}-\textsc{decl} //
\glft `(A respectable person) may have gone.' //
\endgl
\xe
\end{exampleblock}
\begin{itemize}
\item Rich agglutinative morphology
\item Suffixes indicate:
  \begin{itemize}
  \item Subject honorification (-si-)
  \item Past tense (-ess-)
  \item Presumptive mood (-keyss-)
  \item Politeness (-sup-)
  \item Indicative mood (-ni-)
  \item Declarative speech act (-ta)
  \end{itemize}
\end{itemize}
\end{frame}

\section{Word Order}

\begin{frame}
\frametitle{Word Order (slovosled)}
\begin{itemize}
\item SVO (Subject-Verb-Object):
  \begin{itemize}
  \item Most mainland Southeast Asian languages
  \item All Sinitic (Chinese) languages
  \item Examples: Thai, Malay, Vietnamese, Khmer
  \end{itemize}
\item SOV (Subject-Object-Verb):
  \begin{itemize}
  \item Most Tibeto-Burman languages
  \item Korean and Japanese
  \end{itemize}
\item VSO (Verb-Subject-Object):
  \begin{itemize}
  \item Rarer, found in some Austronesian languages
  \item Example: Tagalog (Philippines)
  \end{itemize}
\end{itemize}
\end{frame}



\begin{frame}
\frametitle[allowframebreaks]{SVO Word Order}
\begin{itemize}
\item Most common in mainland Southeast Asian languages
\item Examples: Thai, Vietnamese, Chinese languages
\end{itemize}
\begin{exampleblock}{Vietnamese (Austroasiatic)}
  \ex
  \begingl
\gla Tâm yêu Hiền //
\glb Tam love Hien //
\glft `Tam loves Hien.' //
\endgl
\xe
\end{exampleblock}

\framebreak

\begin{exampleblock}{Cantonese (Sinitic)}
\begingl
\gla Ngo ji dou go daapon //
\glb I know \textsc{cl} answer //
\glft `I know the answer.' //
\endgl
\end{exampleblock}
\begin{exampleblock}{Hmong}
\begingl
\gla Tus dev tom tus npua //
\glb \textsc{clf} dog bite \textsc{clf} pig //
\glft `The dog bit the pig.' //
\endgl
\end{exampleblock}
\end{frame}

\begin{frame}
\frametitle{SOV Word Order}
\begin{itemize}
\item Common in Tibeto-Burman languages, Korean, and Japanese
\item Often features postpositions and sentence-final verbal morphology
\end{itemize}
\begin{exampleblock}{Burmese}
\begingl
\gla Ko The Hlaing-ga akhji-go magazin-dwe po-pe-ba-de //
\glb Ko The Hlaing-\textsc{subj} brother-to magazine-\textsc{pl} send-give-\textsc{pol}-\textsc{real} //
\glft `Ko The Hlaing sent magazines to his brother.' //
\endgl
\end{exampleblock}
\begin{exampleblock}{Korean (Koreanic)}
\begingl
\gla John-i Mary-ege chaek-eul ju-ess-da //
\glb John-\textsc{nom} Mary-\textsc{dat} book-\textsc{acc} give-\textsc{past-decl} //
\glft `John gave Mary the book.' //
\endgl
\end{exampleblock}
\end{frame}

\begin{frame}
\frametitle{VSO Word Order}
\begin{itemize}
\item Rarer in the region, found in some Austronesian languages
\item Example: Tagalog (Philippines)
\end{itemize}
\begin{exampleblock}{Tagalog (Austronesian)}
\begingl
\gla Nakita ng Juan si Maria kahapon //
\glb \textsc{perf}.see \textsc{gen} Juan \textsc{nom} Maria yesterday //
\glft `Juan saw Maria yesterday.' //
\endgl
\end{exampleblock}
\end{frame}

\begin{frame}
\frametitle{Word Order Flexibility}
\begin{itemize}
\item Some languages show flexibility in word order
\item Topic-prominent languages may front important information
\item Example: Japanese (basic SOV, but allows OSV for topicalization)
\end{itemize}
\begin{exampleblock}{Japanese (Japonic)}
\begingl
\gla Sono hon-wa John-ga yonda //
\glb that book-\textsc{top} John-\textsc{nom} read.\textsc{past} //
\glft `That book, John read.' //
\endgl
\end{exampleblock}
\end{frame}




\section{Lexical Tone}

\begin{frame}
\frametitle{Lexical Tone (významotvorný tón)}
\begin{itemize}
\item Many East and Southeast Asian languages use tone to distinguish meaning
\item Case study: Zhongshan Min (China)
  \begin{itemize}
  \item About 150,000 speakers
  \item Complex tonal system with up to 8 tones
  \end{itemize}
\item Tonal categories:
  \begin{itemize}
  \item Historically derived from Middle Chinese tonal categories
  \item Often split into high (yin) and low (yang) registers
  \end{itemize}
\end{itemize}
\end{frame}

\section{Classifier Constructions}

\begin{frame}
\frametitle{Classifier Constructions (numerátory)}
\begin{itemize}
\item Widespread feature in East and Southeast Asian languages
\item Used with numbers and demonstratives
\item Classifiers categorize nouns based on semantic features:
  \begin{itemize}
  \item Shape
  \item Size
  \item Animacy
  \item Function
  \end{itemize}
\item Examples from Malay and Chinese given in the text
\end{itemize}
\end{frame}

\section{Serial Verb Constructions}

\begin{frame}
\frametitle{Serial Verb Constructions (serializace sloves)}
\begin{itemize}
\item Common feature in many Southeast Asian languages
\item Multiple verbs in a single clause without overt conjunction
\item Case study: ACQUIRE verb in Mainland Southeast Asia
  \begin{itemize}
  \item Primary meaning: 'come to have'
  \item Modal meaning: 'can' (in postverbal position)
  \item Resultative meaning
  \item Links temporal, manner, quantifier, and resultative complements
  \end{itemize}
\item Spread across language families due to long-lasting language contact
\end{itemize}
\end{frame}

\section{Multiple Pronouns and Address Systems}

\begin{frame}
\frametitle{Multiple Pronouns and Address Systems}
\begin{itemize}
\item Complex systems of personal pronouns and forms of address
\item Reflect social relationships and contexts
\item Examples:
  \begin{itemize}
  \item Thai: Different forms for politeness, gender, age
  \item Malay: Plain and polite forms
  \item Japanese: Multiple levels of formality
  \end{itemize}
\item Use of kinship terms and names instead of pronouns (e.g., in Malay)
\end{itemize}
\end{frame}

\section{Honorific Forms}

\begin{frame}
\frametitle{Honorific Forms}
\begin{itemize}
\item Elaborate systems in some languages to show respect or social distance
\item Examples:
  \begin{itemize}
  \item Thai nicknames: Based on birth order, animals, fruits, appearance, etc.
  \item Korean: Separate honorific nouns and verbs
    \begin{itemize}
    \item e.g., 'father': abeonim (honorific) vs. abeoji (plain)
    \item e.g., 'to eat': japsusi- (honorific) vs. meok- (plain)
    \end{itemize} 
  \end{itemize}
\end{itemize}
\end{frame}

\begin{frame}
\frametitle{Conclusion}
\begin{itemize}
\item East and Southeast Asian languages show diverse features
\item Some common characteristics:
  \begin{itemize}
  \item Tonal systems
  \item Classifier constructions
  \item Complex pronoun and address systems
  \item Honorific forms
  \end{itemize}
\item Structural convergence due to long-term language contact
\item Rich area for linguistic study and comparison
\end{itemize}
\end{frame}

\end{document}