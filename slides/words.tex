\documentclass{beamer}
\usepackage{fontspec}
\usepackage{xeCJK}
\setCJKmainfont{Noto Sans CJK SC}
\newfontfamily\Libertine[Mapping=tex-text]{Linux Libertine O}
\usetheme{Madrid}
\usecolortheme{crane}

%\usepackage{mygb4e}
\usepackage{langsci-gb4e}
\renewcommand{\eachwordtwo}{\Libertine}
\usepackage[e,j]{mtg2e}
\usepackage{xcolor}
\renewcommand{\mtcitestyle}[1]{\textcolor{teal}{\textit{#1}}}
\newcommand{\msa}{\mtciteform}
\newcommand{\lao}{\mtciteform}
\newcommand{\vie}{\mtciteform}
\newcommand{\cmn}{\mtciteform}
\newcommand{\yue}{\mtciteform}
\newcommand{\ind}{\mtciteform}

\title{Vocabulary}
\author[Francis Bond]{Francis Bond}
\date{2024}\begin{document}

\frame{\titlepage}

\section{Loans as a map of the past}

% Slide 1
\begin{frame}{Loans as Indicators of Cultural History}
    \begin{itemize}
        \item Words in a language can be either indigenous or borrowed from other languages.
        \item Loanwords are linguistic echoes of cultural interactions between different societies.
        \item They provide clues about the relationships, trade, and influences between cultures.
        \item Some loanwords are completely assimilated, while others retain foreign features.
        \item Studying loanwords helps reconstruct the cultural and historical interactions of a society.
    \end{itemize}
\end{frame}

% Slide 2
\begin{frame}{Types of Loanwords}
    \begin{itemize}
        \item \textbf{Indigenized Loanwords:} Words adapted fully to the native language, e.g., \eng{gift}, \eng{root}.
        \item \textbf{Prestige Loanwords:} Borrowed from languages of higher social or political status, e.g., Latin, French.
        \item \textbf{Cultural Loanwords:} Words borrowed to describe new cultural or technological items, e.g., \eng{coffee}, \eng{sushi}.
        \item \textbf{Calques:} Literal translations of foreign expressions, such as \eng{skyscraper} (translated into other languages).
        \item Loanwords often reflect social change, technological advancement, or new cultural influences.
    \end{itemize}
\end{frame}

% Slide 3
\begin{frame}{A Short History of English Loanwords (Early Influence)}
    \begin{itemize}
        \item During the 9th-10th centuries, Old Norse settlers in Britain introduced many words.
        \item Words like \eng{take}, \eng{get}, and \eng{want} became part of everyday English vocabulary.
        \item The similarities between Old Norse and Anglo-Saxon led to seamless borrowing.
        \item These words often refer to basic actions and objects, such as \eng{die}, \eng{hit}, \eng{low}.
        \item Over time, they have become indigenized, and their foreign origins are mostly unrecognized today.
    \end{itemize}
\end{frame}

% Slide 4
\begin{frame}{Norman Conquest and French Influence}
    \begin{itemize}
        \item 1066: Norman Conquest brought extensive French influence on English.
        \item French became the language of the ruling class, while English was spoken by the common people.
        \item Many terms related to governance, law, and nobility were borrowed, e.g., \eng{crown, nation, judge}.
        \item Legal and military terms also entered English, such as \eng{court}, \eng{crime}, \eng{army}, and \eng{navy}.
        \item French influence remained strong for centuries, leading to a dual vocabulary for common vs. formal usage.  Famously \eng{cow} vs \eng{beef}, \eng{sheep} vs \eng{mutton}, \ldots
    \end{itemize}
\end{frame}

% Slide 5
\begin{frame}{Latin and Greek Influence via French}
    \begin{itemize}
        \item The introduction of the printing press in the late 1400s led to an influx of Latin and Greek terms.
        \item Many learned words were borrowed from French but originated in Latin or Greek.
        \item Examples include scholarly and scientific terms such as \eng{scientific}, \eng{describe}, and \eng{bonus}.
        \item Latin had been the common language of learning across Europe during the Middle Ages.
        \item Borrowed affixes, like \eng{ex-} (e.g., \eng{ex-wife}) and \eng{-able} (e.g., \eng{legible}), became productive in English.
    \end{itemize}
\end{frame}

% Slide 6
\begin{frame}{Influence of Global Expansion on English}
    \begin{itemize}
        \item As Britain expanded globally, it acquired words from diverse languages and cultures.
        \item Examples: \eng{pony}, \eng{potato}, and \eng{tomato} from American Indian languages.
        \item \eng{Coffee} was borrowed from Arabic, passing through Turkish before entering English.
        \item Japanese terms like \eng{sushi} and \eng{tsunami} were adopted as global awareness increased.
        \item These loans often indicate exposure to new items, cuisines, and cultural practices.
    \end{itemize}
\end{frame}

% Slide 7
\begin{frame}{Phonological Impact of Loanwords}
    \begin{itemize}
        \item Loanwords can change the phonology of the borrowing language.
        \item The phoneme /v/ became independent in English due to French loans.
        \item In Old English, [v] was an allophone of /f/, e.g., \eng{leaf} vs. \eng{leaves}.
        \item French loans with /v/, such as \eng{veal}, created new phonemic contrasts.
        \item Loanwords can also introduce new phonological features or entire sounds.
    \end{itemize}
\end{frame}

% Slide 8
\begin{frame}{Malay: Historical Loanwords}
    \begin{itemize}
        \item Malay is positioned at a crossroads between India and China, facilitating cultural exchange.
        \item The oldest loans are from the Indic influence, especially Sanskrit and Pali.
        \item Indic influence lasted from the 2nd century BC to the 9th century AD.
        \item Borrowed terms include \msa{rasa} (feel, taste), \msa{nama} (name), and \msa{guru} (teacher).
        \item Indic loans are thoroughly indigenized, and speakers may be unaware of their origins.
    \end{itemize}
\end{frame}

% Slide 9
\begin{frame}{Arabic Influence in Malay}
    \begin{itemize}
        \item Arabic influence arrived with the spread of Islam in the 1400s.
        \item Many Arabic loans relate to religion, law, and social practices.
        \item Examples include \msa{fikir} (think), \msa{lihat} (see), and \msa{haram} (forbidden).
        \item The influence of Arabic remains strong in both Indonesian and Malaysian vocabulary.
        \item Arabic terms often retain distinct phonological features, such as fricatives
        \item Arabic terms often have distinctive spellling, like \msa{kh}
    \end{itemize}
\end{frame}

% Slide 10
\begin{frame}{Malay Loanwords from European Languages}
    \begin{itemize}
        \item After the fall of Malacca in 1511, European influence intensified in the Malay world.
        \item Portuguese loans include \msa{gereja} (church), \msa{garpu} (fork), and \msa{almari} (cupboard).
        \item Dutch loans include \msa{senapang} (gun) and \msa{pelakat} (gum).
        \item Later, English words were borrowed during the British colonial period, e.g., \msa{tiket} (ticket), \msa{kamera} (camera).
        \item These loans reflect technological and cultural changes brought by European colonization.
        \item Indonesian borrows more from Dutch and Malay from English (their respective colonisers.
          \\ E.g., \msa{kantor} vs \msa{ofis} ``office'' or \msa{karcis} ($<$\eng{karrtjes} pl) vs \msa{tiket} ``tiket''
    \end{itemize}
\end{frame}

% Slide 11
\begin{frame}{Malay Loanwords from Closely Related Languages}
    \begin{itemize}
        \item Malay has also borrowed from Tamil, Chinese, and other regional languages.
        \item Tamil loans include \msa{kedai} (shop), \msa{katil} (bed), and \msa{mangga} (mango).
        \item Chinese loans include \msa{teh} (tea) and \msa{mi} (noodles).
        \item These loans are often related to trade, food, and cultural exchanges.
        \item Many of these terms are fully integrated into the Malay lexicon.
    \end{itemize}
\end{frame}

% Slide 12
\begin{frame}{Mainland Southeast Asia: Indic and Native Vocabulary}
    \begin{itemize}
        \item Mainland Southeast Asia was influenced by Indian civilization, especially Theravada Buddhism.
        \item Languages such as Burmese, Thai, Lao, and Khmer borrowed from Sanskrit and Pali (Indic).
        \item Indic loans often serve formal or literary purposes, distinguishing them from native terms.
        \item Examples:          \begin{tabular}[t]{lll}
            Indic & Native Thai & Gloss \\
            \eng{sı̌isà'} &  \eng{hǔa} & head \\
            \eng{phranêet} & \eng{taa} & eye \\
               \eng{sunák} & \eng{mǎa} & dog\\
          \end{tabular}
        \item Indic influence persists in the scripts and learned vocabulary
        \item New vocabulary is often made through calques
          \\ e.g. \eng{thoola².that¹} ``television [far.vision]''
        \item Borrowing from Europe is less common (although becoming so)
    \end{itemize}
\end{frame}

% Slide 13
\begin{frame}{Mainland Southeast Asia: Chinese Influence}
    \begin{itemize}
        \item Chinese influence in mainland Southeast Asia extends back over 1,000 years.
        \item Chinese loans are present in Lao, Thai, and Vietnamese, especially for numbers and classifiers.
        \item Examples of Lao loans from Chinese: \lao{toq2} (table), \lao{ngen2} (money), \lao{bòò1} (not).
        \item Chinese influence is especially strong in Vietnamese, where many basic nouns are borrowed.
        \item Chinese cultural influence has shaped both the lexicon and grammar of these languages.
    \end{itemize}
\end{frame}

% Slide 14
\begin{frame}{Chinese Influence on Korean and Japanese}
    \begin{itemize}
        \item Chinese influence on Korean and Japanese began during the first millennium AD.
        \item Korean and Japanese borrowed heavily from Chinese, especially religious and cultural terms.
        \item Historic date: 552 AD, when Buddhism was officially adopted in Japan.
        \item Examples of Chinese loans in Japanese: \jpn{Shaka} (Buddha), \jpn{hachi} (bowl), \jpn{sugi} (cedar).
        \item Chinese influence extends beyond vocabulary to include script, culture, and social norms.
    \end{itemize}
\end{frame}

% Slide 15
\begin{frame}{European Loans in Japanese}
    \begin{itemize}
        \item In the mid-1500s, Portuguese traders introduced words like \jpn{tempura}.
        \item Dutch loans entered during the Edo period, e.g., \jpn{kokku} (cook), \jpn{garasu} (glass).
        \item French influence appeared in the 18th century, e.g., \jpn{atorie} (artist's studio).
        \item Post-Meiji Restoration, many English words were borrowed, e.g., \jpn{terebi} (TV), \jpn{sararii} (salary).
        \item Foreign loans often indicate technological advancement or cultural shifts.
    \end{itemize}
\end{frame}

% Slide 16
\begin{frame}{Foreign Loans in Contemporary Japanese}
    \begin{itemize}
        \item Modern Japanese has borrowed extensively from English and other European languages.
        \item Loans include terms from technology, fashion, and popular culture, e.g., \jpn{fakkusu} (fax).
        \item Many loans represent items associated with Western culture, e.g., \jpn{beddo} (Western-style bed).
        \item Hybrid words are common, e.g., \jpn{ichigo-ēdo} (strawberry + "ade") and \jpn{wan-man-kaa} (one-man car).
        \item The use of loans in Japanese reflects a dynamic interplay of traditional and modern influences.
    \end{itemize}
\end{frame}

\section{Building words --- Derivational Morphology}


% Slide 1
\begin{frame}{Word Structure: Derivational Morphology}
    \begin{itemize}
        \item Morphology refers to the study of word structure and formation.
        \item East and Southeast Asian languages generally lack inflection.
        \item Inflection changes a word's form based on grammatical context, e.g., tense or number.
        \item Derivation involves creating new words through affixation, compounding, or reduplication.
        \item Derivational morphology is more common than inflectional morphology in many Asian languages.
    \end{itemize}
\end{frame}

% Slide 2
\begin{frame}{Derivational Processes in English}
    \begin{itemize}
        \item Derivation involves adding prefixes or suffixes to create new words.
        \item Examples of prefixes: \eng{un-} (e.g., \eng{unreliable}), \eng{re-} (e.g., \eng{rewritten}).
        \item Examples of suffixes: \eng{-able} (e.g., \eng{reliable}), \eng{-ness} (e.g., \eng{sadness}).
        \item Derivational affixes can be productive, meaning they can be applied to new words.
        \item Compounding (e.g., \eng{bookshop}, \eng{blackboard}) and reduplication (e.g., \eng{choo-choo}) are other derivational processes.
    \end{itemize}
\end{frame}

% Slide 3
\begin{frame}{Compounding in East and Southeast Asian Languages}
    \begin{itemize}
        \item Sinitic and mainland Southeast Asian languages rely heavily on compounding.
        \item Compounds often consist of two elements, forming nouns, adjectives, or verbs.
        \item Compounding is a productive process and forms many new words in these languages.
        \item Compounds can have meanings not entirely predictable from their components.
    \end{itemize}
\end{frame}

% Slide 4
\begin{frame}{Compounding in Mandarin Chinese}
    \begin{itemize}
        \item Mandarin Chinese compounding includes both lexicalized and productive forms.
        \item Some compounds are parallel, where neither element modifies the other, e.g., \cmn{huā-mù} (flower-tree) meaning "vegetation".
        \item Metaphorical compounds are also common, e.g., \cmn{gǒu-xióng} (dog-bear) meaning "bear".
        \item Compound verbs can express a combination of similar or related actions, e.g., \cmn{tòng-kǔ} (painful-bitter) meaning "bitter and painful".
        \item Compounding is a key process in forming Mandarin nouns and verbs.
    \end{itemize}
\end{frame}

% Slide 5
\begin{frame}{Productive Compounding in Mandarin}
    \begin{itemize}
        \item New compound nouns can be created easily using productive patterns.
        \item Semantic relationships in compounding include location, purpose, and material.
        \item Example: \cmn{táidēng} (table lamp) denotes a place where N2 is used.
        \item The flexibility of compounding allows for the formation of many complex terms.
        \item Compounding reflects the isolating nature of Mandarin, with minimal inflection.
    \end{itemize}
\end{frame}

% Slide 6
\begin{frame}{Compounding in Japanese}
    \begin{itemize}
        \item Japanese compounding is the most productive method for creating new words.
        \item Compounds can involve native Japanese, Sino-Japanese (S-J), or Foreign elements.
        \item Examples: \jpn{aki-sora} (autumn-sky) meaning "autumn sky", \jpn{to-kei} (time-meter) meaning "clock".
        \item Verbal nouns and adjective-verb combinations are also common in Japanese.
        \item S-J compounds often relate to technical terms, e.g., \jpn{syakai-gaku} (sociology).
    \end{itemize}
\end{frame}

% Slide 7
\begin{frame}{Compound Verbs in Japanese}
    \begin{itemize}
        \item Compound verbs can express the manner or aspect of an action.
        \item Example: \jpn{naguri-korosu} (beat-kill) meaning "kill by beating".
        \item Compounds with aspectual modification include \jpn{kaki-ageru} (write-up).
        \item Native and S-J compounds follow different element orders due to OV vs. VO patterns.
        \item Compounding in Japanese reflects a combination of native structure and Chinese influence.
    \end{itemize}
\end{frame}

% Slide 8
\begin{frame}{Extended Meanings in Japanese Compounds}
    \begin{itemize}
        \item Compounds can acquire extended meanings beyond their literal sense.
        \item Example: \jpn{hito-goroshi} (person-kill) can mean both ``killing a person" and ``a killer".
        \item Compound meanings can shift from action to agent, instrument, or location.
        \item Example: \jpn{ha-migaki} (tooth-polishing) meaning ``toothpaste".
        \item Such extensions reflect cultural nuances and historical language use.
    \end{itemize}
\end{frame}

% Slide 9
\begin{frame}{Abbreviation and Blending in Japanese}
    \begin{itemize}
        \item Abbreviation is a productive process in Japanese, reducing phrases to shorter forms.
        \item Examples: \jpn{paasonaru konpyutaa} (personal computer) to \jpn{paso-kon}.
        \item Abbreviation often targets lengthy compounds and adapts them to preferred syllable counts.
        \item Blending can create new words by merging elements from different words.
        \item Abbreviated words are common in media and everyday communication.
    \end{itemize}
\end{frame}

% Slide 10
\begin{frame}{Reduplication in East and Southeast Asian Languages}
    \begin{itemize}
        \item Reduplication is more common in East and Southeast Asian languages than in English.
        \item Vietnamese uses reduplication to convey meanings like attenuation or intensification.
        \item Example: \vie{trắng trắng} meaning "be whitish".
        \item Thai and Indonesian also use reduplication for plurality, e.g., \textit{dèk-dèk} (children).
        \item Reduplication may involve partial repetition or phonological changes.
    \end{itemize}
\end{frame}

% Slide 11
\begin{frame}{Rhyming and Chiming Reduplication}
    \begin{itemize}
        \item Indonesian features rhyming and chiming reduplication, similar to \eng{lovey-dovey} in English.
        \item Rhyming reduplication emphasizes or intensifies meanings, e.g., \ind{kaya-raya} (rich).
        \item Chiming reduplication involves varying vowels, e.g., \ind{teki-teka} (riddle).
        \item These forms are often idiomatic and carry cultural or emotional connotations.
        \item Extensive reduplication is also common in Thai and Vietnamese, reflecting emphasis or variety.
    \end{itemize}
\end{frame}

% Slide 12
\begin{frame}{Derivational Affixation in Sinitic Languages}
    \begin{itemize}
        \item Sinitic languages like Cantonese have some derivational affixes, though limited.
        \item Prefixes and suffixes can form nouns from other words, e.g., \yue{góng-faat} (speak-way).
        \item The suffix \textit{-fa} forms causative verbs, e.g., \yue{yihndoih-fa} (modernize).
        \item Hakka uses suffixes to indicate gender, derived from kin terms.
        \item Though affixation is not prominent, it still contributes to word formation.
    \end{itemize}
\end{frame}

% Slide 13
\begin{frame}{Derivational Affixation in Mainland Southeast Asia}
    \begin{itemize}
        \item Thai and Vietnamese use derivational affixes to form new words and modify meanings.
        \item Thai prefix \textit{kaan-} forms abstract nouns, e.g., \textit{kaan-lên} (playing).
        \item Vietnamese uses prefixes like \vie{phản-} (counter, anti-: from Chinese 反), e.g., \textit{phản-kháng} (protest).
        \item Many affixes in Vietnamese are of Chinese origin and used in educated contexts.
        \item These affixes are more productive in formal registers, such as journalism and academia.
    \end{itemize}
\end{frame}

% Slide 14
\begin{frame}{Productive Derivation in Austronesian Languages}
    \begin{itemize}
        \item Austronesian languages like Malay have extensive derivational morphology.
        \item Malay uses prefixes, suffixes, and circumfixes to derive new words.
        \item Example: Prefix \ind{meN-} forms dynamic verbs, e.g., \ind{men-darat} (go ashore).
        \item Suffix \ind{-kan} derives transitive verbs, e.g., \ind{mem-bersih-kan} (to clean).
        \item Derivational morphology is highly productive and central to word formation.
    \end{itemize}
\end{frame}

% Slide 15
\begin{frame}{Derivational Affixes in Korean and Japanese}
    \begin{itemize}
        \item Korean and Japanese also use derivational affixes, including productive causative forms.
        \item Korean causative suffix \textit{-i} has several allomorphs, e.g., \textit{boi-} (to show).
        \item Japanese causative suffix \textit{-(sa)se} is also highly productive, e.g., \textit{misase-} (cause to look at).
        \item Irregular derivational affixes exist but are less productive.
        \item Derivational affixation contributes to verb formation and modified meanings in both languages.
    \end{itemize}
\end{frame}

% Slide 16
\begin{frame}{Summary: Derivational Morphology in East and Southeast Asia}
    \begin{itemize}
        \item East and Southeast Asian languages show diversity in derivational morphology.
        \item Compounding, reduplication, and affixation are used to create new words.
        \item Sinitic and mainland Southeast Asian languages tend to prefer compounding and reduplication.
        \item Austronesian languages exhibit extensive and productive affixation.
        \item Derivational morphology is a key feature in understanding how words are formed across different language families.
    \end{itemize}
\end{frame}

\section{Meaning differences between languages}

% Slide 1
\begin{frame}{Meaning Differences Between Languages}
    \begin{itemize}
        \item Every language has unique words for culturally specific items.
        \item These words may lack direct equivalents in other languages.
        \item Examples include \textit{sake} (Japanese rice alcohol) and \textit{omiai} (formal marriage meeting).
          \item  \textit{sake} then broadened to mean any alcohol (polysemous with Japanese alcohol) with  \textit{nihonshu} meaning just  \textit{sake}!  
        \item Cultural differences also affect words for emotions, values, and life ideals.
        \item Even seemingly basic items and actions can vary significantly between languages.
    \end{itemize}
\end{frame}

% Slide 2
\begin{frame}{Challenges in Describing Word Meanings}
    \begin{itemize}
        \item Describing word meanings is a complex task.
        \item Definitions must use simpler, more understandable terms.
        \item Circular and obscure definitions lead to confusion and inaccuracy.
        \item Polysemy refers to a word having multiple interrelated meanings.
        \item Understanding polysemy is crucial for comparing meanings across languages.
    \end{itemize}
\end{frame}

% Slide 3
\begin{frame}{Different Patterns of Polysemy}
    \begin{itemize}
        \item Words can have different patterns of polysemy across languages.
        \item Russian \textit{nos} and Japanese \textit{hana} both mean "nose" and can also mean "beak" or "trunk."
        \item These words are polysemous, having multiple meanings that overlap with English \textit{nose}.
        \item English \textit{live} has two meanings: "to exist" and "to reside somewhere."
        \item In Malay, these meanings are split into two words: \textit{hidup} and \textit{tinggal}.
    \end{itemize}
\end{frame}

% Slide 4
\begin{frame}{Indicators of Polysemy}
    \begin{itemize}
        \item Different grammatical properties can indicate polysemy.
        \item English \textit{live} in "reside somewhere" requires a place expression.
        \item Malay \textit{buat} means both "do" and "make," depending on context.
        \item The presence of different constraints or arguments can reveal distinct meanings.
        \item Understanding polysemy helps clarify seemingly ambiguous language use.
    \end{itemize}
\end{frame}

% Slide 5
\begin{frame}{Existence and Possession: A Common Polysemy}
    \begin{itemize}
        \item Many languages have a single word for "existence" and "have."
        \item Malay \textit{ada} and Mandarin \textit{yǒu} mean both "there is" and "have."
        \item The different meanings are distinguished by their grammatical frames.
        \item "There is" takes one argument; "have" takes two arguments.
        \item Example: Malay \textit{ada dua ekor lembu} (there are two cows) vs. \textit{Orang ini ada dua ekor lembu} (this person has two cows).
    \end{itemize}
\end{frame}

% Slide 6
\begin{frame}{Different Meanings for Basic Items}
    \begin{itemize}
        \item Some languages lack a word that exactly matches English "water."
        \item Japanese has \textit{mizu} for "non-hot water" and \textit{yu} for "hot water."
        \item \textit{Mizu} cannot be used for hot water, and combining \textit{atsui} ("hot") with \textit{mizu} sounds unnatural.
        \item This distinction is ignored in English, which uses "water" for both.
        \item Different lexical distinctions reflect cultural perceptions of basic items.
    \end{itemize}
\end{frame}

% Slide 7
\begin{frame}{Different Meanings for Actions}
    \begin{itemize}
        \item Many Asian languages differentiate types of "breaking" actions.
        \item Malay has \textit{putus} ("break in two"), \textit{patah} ("break but not sever"), and \textit{pecah} ("break into pieces").
        \item Cantonese uses specific verbs based on how something breaks, e.g., \textit{dá laahn} ("smash into pieces").
        \item Japanese uses different words for "break off" (\textit{oru}), "destroy" (\textit{kowasu}), and "cut off" (\textit{kiru}).
        \item English has a general term "break" without such fine distinctions.
    \end{itemize}
\end{frame}

% Slide 8
\begin{frame}{Differences in Verbs Like "Come"}
    \begin{itemize}
        \item English \textit{come} can indicate movement towards the speaker, addressee, or a third person.
        \item Japanese \textit{kuru}, Malay \textit{datang}, and Korean \textit{oda} have more restricted usage.
        \item Japanese does not allow shifting the point of view as English \textit{come} does.
        \item Instead, Japanese uses a verb equivalent to "go" in such contexts.
        \item These differences highlight subtle semantic distinctions in movement verbs.
    \end{itemize}
\end{frame}

% Slide 9
\begin{frame}{Culturally Based Specialization in the Lexicon}
    \begin{itemize}
        \item Some meaning differences are culturally motivated, reflecting lifestyle and values.
        \item Asian languages often have different words for rice in different forms.
        \item Example: Malay \textit{padi} (growing rice), \textit{beras} (raw rice), \textit{nasi} (cooked rice).
        \item The word for cooked rice often doubles as a term for a meal, e.g., Japanese \textit{gohan} and Cantonese \textit{sihk faahn}.
        \item Language reflects the cultural importance of rice in Asian societies.
    \end{itemize}
\end{frame}

% Slide 10
\begin{frame}{Kinship Terminology and Cultural Concerns}
    \begin{itemize}
        \item Kinship terms often reflect cultural norms and family hierarchy.
        \item East and Southeast Asian languages distinguish between older and younger siblings.
        \item Thai has words for "older sibling" and "younger sibling" without gender distinction.
        \item Mandarin has distinct terms for "older brother," "older sister," "younger brother," and "younger sister."
        \item The relative age of siblings establishes lifelong social seniority in many Asian cultures.
    \end{itemize}
\end{frame}

% Slide 11
\begin{frame}{Order of Birth and Specific Kinship Terms}
    \begin{itemize}
        \item Some languages specify the exact order of birth in kinship terms.
        \item Mandarin has expressions for "first brother," "second brother," etc.
        \item Malay uses nicknames like \textit{long} ("oldest sibling") and \textit{teh} ("fourth sibling").
        \item These distinctions emphasize the importance of family roles.
        \item Specific kinship terms reflect cultural emphasis on family hierarchy and lineage.
    \end{itemize}
\end{frame}

% Slide 12
\begin{frame}{Terms for Emotions and Attitudes}
    \begin{itemize}
        \item Terms for emotions and attitudes vary widely across cultures.
        \item The "areal lexicon" of mainland Southeast Asia includes many complex expressions for feelings.
        \item Matisoff's concept of "psycho-collocation" involves terms built around "heart" or "spirit."
        \item Languages like Thai, Malay, and Lai have dozens of psycho-collocations.
        \item These expressions provide a nuanced way to discuss emotions and mental states.
    \end{itemize}
\end{frame}

% Slide 13
\begin{frame}{Examples of Psycho-Collocations}
    \begin{itemize}
        \item Psycho-collocations use a noun like "heart" combined with adjectives or verbs.
        \item Lai: \textit{thin haaN} ("liver become liquid") meaning "angry."
        \item Thai: \textit{thùuk-cay} ("correct-heart") meaning "please, satisfy."
        \item Malay: \textit{panas hati} ("hot liver") meaning "angry, worked up."
        \item These terms provide speakers with a rich emotional vocabulary for everyday conversation.
    \end{itemize}
\end{frame}

% Slide 14
\begin{frame}{Use of Psycho-Collocations in Daily Language}
    \begin{itemize}
        \item Psycho-collocations are central to expressing emotions in many Asian languages.
        \item Lai and Thai conversations frequently use these terms for emotional precision.
        \item Expressions involving "heart" or "liver" can describe a wide range of feelings.
        \item Psycho-collocations are often more common than simple one-word emotion terms.
        \item They allow speakers to convey subtle differences in mental and emotional states.
    \end{itemize}
\end{frame}

% Slide 15
\begin{frame}{Specialized Vocabulary for Nature and Agriculture}
    \begin{itemize}
        \item Many Asian languages have rich vocabularies for nature and agriculture.
        \item Terms for weather, plants, insects, and fish reflect the traditional lifestyle.
        \item Japanese has an extensive vocabulary for natural beauty, seasons, and related expressions.
        \item These words are culturally important in Japanese poetry and literature.
        \item The vocabulary reflects a cultural appreciation of nature and seasonal changes.
    \end{itemize}
\end{frame}

% Slide 16
\begin{frame}{Summary: Cultural and Lexical Differences Between Languages}
    \begin{itemize}
        \item Words for culturally specific items often lack direct equivalents in other languages.
        \item Even basic words can have different meanings and lexical distinctions.
        \item Cultural factors shape specialized vocabularies, such as terms for family, emotions, and nature.
        \item Psycho-collocations and kinship terms highlight cultural norms and values.
        \item Understanding lexical differences is key to understanding cultural diversity in language.
    \end{itemize}
\end{frame}


\section{Some cultural key words}

% Slide 1
\begin{frame}{Cultural Key Words}
    \begin{itemize}
        \item Cultural key words are highly salient and deeply culture-laden.
        \item They act as focal points around which cultural domains are organized.
        \item Examples: \textit{love}, \textit{honesty}, and \textit{mate} (Australian).
        \item Cultural key words often appear in phrases, proverbs, and popular sayings.
        \item Analyzing these words can reveal core organizing principles of a culture.
    \end{itemize}
\end{frame}

% Slide 2
\begin{frame}{Identifying Cultural Key Words}
    \begin{itemize}
        \item Key words are usually common within specific domains (e.g., emotions, social values).
        \item They often appear frequently in conversation and explanations.
        \item Lack of equivalents in other languages can indicate a cultural key word.
        \item Example: Japanese \textit{omiai} (formal meeting of potential marriage partners).
        \item Key words provide insights into the underlying cultural values.
    \end{itemize}
\end{frame}

% Slide 3
\begin{frame}{Challenges in Capturing Meanings}
    \begin{itemize}
        \item Describing meanings must avoid culture bias (ethnocentrism).
        \item Using English-specific concepts can distort meanings in other languages.
        \item Simpler meanings are more likely to be shared across languages.
        \item Semantic primes are basic "atoms of meaning" present in all languages.
        \item Examples of semantic primes: \textit{someone}, \textit{think}, \textit{good}, \textit{because}.
    \end{itemize}
\end{frame}

% Slide 4
\begin{frame}{The Natural Semantic Metalanguage Approach}
    \begin{itemize}
        \item Semantic primes can be used to describe cultural key words without bias.
        \item Semantic primes help create universally understandable descriptions.
        \item This approach allows meanings to be tested against native speaker intuitions.
        \item Example: The meaning of \textit{malu} (Malay) captured in semantic primes.
        \item This method enables a clearer cross-cultural understanding of key concepts.
    \end{itemize}
\end{frame}

% Slide 5
\begin{frame}{Key Words in Malay: \textit{malu}}
    \begin{itemize}
        \item \textit{Malu} roughly translates to "shame," "shy," or "embarrassed."
        \item It is linked to a fear of others thinking or saying something bad about oneself.
        \item Feeling \textit{malu} often leads to a desire to avoid others.
        \item Examples: \textit{"Aku malu"} ("I'm ashamed"), \textit{"Eh! Malu pulak dia!"} ("She's shy!").
        \item The concept encourages social conformity and respect for others' opinions.
    \end{itemize}
\end{frame}

% Slide 6
\begin{frame}{Semantic Explication of \textit{malu}}
    \begin{itemize}
        \item X feels something bad because they think: "People can know something about me."
        \item "People can think or say something bad about me because of this."
        \item "I don't want this to happen."
        \item "I don't want to be near other people because of this."
        \item "It is good if people can feel something like this."
    \end{itemize}
\end{frame}

% Slide 7
\begin{frame}{Key Words in Malay: \textit{sabar}}
    \begin{itemize}
        \item \textit{Sabar} translates loosely to "patience" but with a broader meaning.
        \item It involves staying calm and enduring suffering without losing control.
        \item Common advice: \textit{"Bersabar..."} ("Be patient"), \textit{"Sabar, jangan menangis"} ("Calm down, don't cry").
        \item Reflects religious ideals in Islam: \textit{"Allah sentiasa bersama dengan orang-orang yang sabar"}.
        \item \textit{Sabar} is encouraged in challenging situations, showing a deep moral value.
    \end{itemize}
\end{frame}

% Slide 8
\begin{frame}{Semantic Explication of \textit{sabar}}
    \begin{itemize}
        \item X felt something bad but did not want to act on it immediately.
        \item X thought: "I don't want to think like this."
        \item X stayed calm and did not act impulsively.
        \item It is good if a person can be like this.
        \item \textit{Sabar} requires inner discipline and self-control.
    \end{itemize}
\end{frame}

% Slide 9
\begin{frame}{Key Words in Chinese: \textit{xiào}}
    \begin{itemize}
        \item \textit{Xiào} represents filial piety or devotion to one's parents.
        \item It emphasizes duty, respect, and caring for parents in old age.
        \item Fixed phrases: \textit{"bǎi shàn xiào wéi xiān"} ("Of the hundred good deeds, \textit{xiào} comes first").
        \item \textit{Xiào} involves remembering and honoring parents and ancestors.
        \item Example: \textit{"Fùmǔ eng bı́ shān gāo, bı́ hǎi shēn"} ("What our parents give us is higher than a mountain, deeper than the ocean").
    \end{itemize}
\end{frame}

% Slide 10
\begin{frame}{Semantic Explication of \textit{xiào}}
    \begin{itemize}
        \item "These people are my father and mother. I exist because of them."
        \item "I want them to feel something very good at all times."
        \item "I have to do many good things for them because of this."
        \item "I don't want them to feel anything bad at any time."
        \item "It is good if a person thinks and acts like this."
    \end{itemize}
\end{frame}

% Slide 11
\begin{frame}{Key Words in Chinese: \textit{rěn}}
    \begin{itemize}
        \item \textit{Rěn} means perseverance or patience, often in difficult situations.
        \item Reflects a Chinese value of long-term goals over immediate desires.
        \item Common phrases: \textit{"rěn qı̀ tūn shēng"} ("swallow one's anger").
        \item \textit{Rěn} involves enduring hardship without complaint for a greater purpose.
        \item It is seen as a source of inner strength and moral virtue.
    \end{itemize}
\end{frame}

% Slide 12
\begin{frame}{Semantic Explication of \textit{rěn}}
    \begin{itemize}
        \item X wants something to happen and knows it requires sustained effort.
        \item X thinks: "I must keep doing these things even if it takes a long time."
        \item If X feels something bad, they still continue.
        \item "It is good if a person always thinks and acts like this."
        \item \textit{Rěn} embodies the importance of endurance and resilience.
    \end{itemize}
\end{frame}

% Slide 13
\begin{frame}{Key Words in Japanese: \textit{amae}}
    \begin{itemize}
        \item \textit{Amae} refers to a sense of dependence or indulgence in a close relationship.
        \item Often linked to the infant's dependency on the mother.
        \item It implies taking the other person's love and care for granted.
        \item Example: \textit{"When I am with Y, nothing bad can happen to me."}
        \item \textit{Amae} is seen as an integral part of Japanese social relationships.
    \end{itemize}
\end{frame}

% Slide 14
\begin{frame}{Semantic Explication of \textit{amae}}
    \begin{itemize}
        \item X thinks about Y: "Y feels something good about me and wants to do good things for me."
        \item "When I am with Y, nothing bad can happen to me."
        \item "I don't have to do anything because of this."
        \item X feels something good because of these thoughts.
        \item \textit{Amae} reflects a positive dependence in relationships.
    \end{itemize}
\end{frame}

% Slide 15
\begin{frame}{Key Words in Japanese: \textit{omoiyari}}
    \begin{itemize}
        \item \textit{Omoiyari} means empathy, consideration, or sensitivity to others' feelings.
        \item It involves anticipating others' needs without being told.
        \item Example: \textit{"Omoiyari no aru ko ni nattene"} ("Please become a person with \textit{omoiyari}").
        \item Reflects the Japanese value of maintaining harmony and avoiding conflict.
        \item \textit{Omoiyari} is a highly valued trait in Japanese social interactions.
    \end{itemize}
\end{frame}

% Slide 16
\begin{frame}{Semantic Explication of \textit{omoiyari}}
    \begin{itemize}
        \item X often thinks: "I can know what this person feels and wants."
        \item "I want to do something good for this person because of this."
        \item "This person doesn't have to say anything."
        \item X acts on this understanding, anticipating needs and desires.
        \item \textit{Omoiyari} embodies the ideal of empathy in Japanese culture.
    \end{itemize}
\end{frame}


\begin{frame}{Conclusion}
  
\end{frame}

\end{document}
